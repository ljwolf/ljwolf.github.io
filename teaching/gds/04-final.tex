\documentclass[]{tufte-handout}

% ams
\usepackage{amssymb,amsmath}

\usepackage{ifxetex,ifluatex}
\usepackage{fixltx2e} % provides \textsubscript
\ifnum 0\ifxetex 1\fi\ifluatex 1\fi=0 % if pdftex
  \usepackage[T1]{fontenc}
  \usepackage[utf8]{inputenc}
\else % if luatex or xelatex
  \makeatletter
  \@ifpackageloaded{fontspec}{}{\usepackage{fontspec}}
  \makeatother
  \defaultfontfeatures{Ligatures=TeX,Scale=MatchLowercase}
  \makeatletter
  \@ifpackageloaded{soul}{
     \renewcommand\allcapsspacing[1]{{\addfontfeature{LetterSpace=15}#1}}
     \renewcommand\smallcapsspacing[1]{{\addfontfeature{LetterSpace=10}#1}}
   }{}
  \makeatother

\fi

% graphix
\usepackage{graphicx}
\setkeys{Gin}{width=\linewidth,totalheight=\textheight,keepaspectratio}

% booktabs
\usepackage{booktabs}

% url
\usepackage{url}

% hyperref
\usepackage{hyperref}

% units.
\usepackage{units}


\setcounter{secnumdepth}{-1}

% citations


% pandoc syntax highlighting
\usepackage{color}
\usepackage{fancyvrb}
\newcommand{\VerbBar}{|}
\newcommand{\VERB}{\Verb[commandchars=\\\{\}]}
\DefineVerbatimEnvironment{Highlighting}{Verbatim}{commandchars=\\\{\}}
% Add ',fontsize=\small' for more characters per line
\newenvironment{Shaded}{}{}
\newcommand{\AlertTok}[1]{\textcolor[rgb]{1.00,0.00,0.00}{\textbf{#1}}}
\newcommand{\AnnotationTok}[1]{\textcolor[rgb]{0.38,0.63,0.69}{\textbf{\textit{#1}}}}
\newcommand{\AttributeTok}[1]{\textcolor[rgb]{0.49,0.56,0.16}{#1}}
\newcommand{\BaseNTok}[1]{\textcolor[rgb]{0.25,0.63,0.44}{#1}}
\newcommand{\BuiltInTok}[1]{#1}
\newcommand{\CharTok}[1]{\textcolor[rgb]{0.25,0.44,0.63}{#1}}
\newcommand{\CommentTok}[1]{\textcolor[rgb]{0.38,0.63,0.69}{\textit{#1}}}
\newcommand{\CommentVarTok}[1]{\textcolor[rgb]{0.38,0.63,0.69}{\textbf{\textit{#1}}}}
\newcommand{\ConstantTok}[1]{\textcolor[rgb]{0.53,0.00,0.00}{#1}}
\newcommand{\ControlFlowTok}[1]{\textcolor[rgb]{0.00,0.44,0.13}{\textbf{#1}}}
\newcommand{\DataTypeTok}[1]{\textcolor[rgb]{0.56,0.13,0.00}{#1}}
\newcommand{\DecValTok}[1]{\textcolor[rgb]{0.25,0.63,0.44}{#1}}
\newcommand{\DocumentationTok}[1]{\textcolor[rgb]{0.73,0.13,0.13}{\textit{#1}}}
\newcommand{\ErrorTok}[1]{\textcolor[rgb]{1.00,0.00,0.00}{\textbf{#1}}}
\newcommand{\ExtensionTok}[1]{#1}
\newcommand{\FloatTok}[1]{\textcolor[rgb]{0.25,0.63,0.44}{#1}}
\newcommand{\FunctionTok}[1]{\textcolor[rgb]{0.02,0.16,0.49}{#1}}
\newcommand{\ImportTok}[1]{#1}
\newcommand{\InformationTok}[1]{\textcolor[rgb]{0.38,0.63,0.69}{\textbf{\textit{#1}}}}
\newcommand{\KeywordTok}[1]{\textcolor[rgb]{0.00,0.44,0.13}{\textbf{#1}}}
\newcommand{\NormalTok}[1]{#1}
\newcommand{\OperatorTok}[1]{\textcolor[rgb]{0.40,0.40,0.40}{#1}}
\newcommand{\OtherTok}[1]{\textcolor[rgb]{0.00,0.44,0.13}{#1}}
\newcommand{\PreprocessorTok}[1]{\textcolor[rgb]{0.74,0.48,0.00}{#1}}
\newcommand{\RegionMarkerTok}[1]{#1}
\newcommand{\SpecialCharTok}[1]{\textcolor[rgb]{0.25,0.44,0.63}{#1}}
\newcommand{\SpecialStringTok}[1]{\textcolor[rgb]{0.73,0.40,0.53}{#1}}
\newcommand{\StringTok}[1]{\textcolor[rgb]{0.25,0.44,0.63}{#1}}
\newcommand{\VariableTok}[1]{\textcolor[rgb]{0.10,0.09,0.49}{#1}}
\newcommand{\VerbatimStringTok}[1]{\textcolor[rgb]{0.25,0.44,0.63}{#1}}
\newcommand{\WarningTok}[1]{\textcolor[rgb]{0.38,0.63,0.69}{\textbf{\textit{#1}}}}

% longtable
\usepackage{longtable,booktabs}

% multiplecol
\usepackage{multicol}

% strikeout
\usepackage[normalem]{ulem}

% morefloats
\usepackage{morefloats}


% tightlist macro required by pandoc >= 1.14
\providecommand{\tightlist}{%
  \setlength{\itemsep}{0pt}\setlength{\parskip}{0pt}}

% title / author / date
\title{04-final}
\date{}


\begin{document}

\maketitle




Your friend from school has just returned from their gap year as a
\href{https://wwoof.org.uk/}{WWOOF-er} in a small estate outside of
Lyon. They claim that there's nothing quite like a French wine. They're
better, more expensive, and you know it!

So, you, now a freshly minted data scientist, decide to moonlight as a
data sommelier. You find a nice dataset of
\href{https://www.kaggle.com/zynicide/wine-reviews}{all the wine
ratings} you can find, and get to work.

\begin{Shaded}
\begin{Highlighting}[]
\KeywordTok{library}\NormalTok{(tidyverse)}
\NormalTok{wines =}\StringTok{ }\KeywordTok{read_csv}\NormalTok{(}\StringTok{'http://ljwolf.org/teaching/gds/wines.csv'}\NormalTok{)}
\NormalTok{wines }\OperatorTok\StringTok{ }\KeywordTok{head}\NormalTok{(}\DecValTok{1}\NormalTok{) }\OperatorTok\StringTok{ }\NormalTok{knitr}\OperatorTok{::}\KeywordTok{kable}\NormalTok{()}
\end{Highlighting}
\end{Shaded}

\begin{longtable}[]{@{}rlllrrlllll@{}}
\toprule
\begin{minipage}[b]{0.00\columnwidth}\raggedleft
X1\strut
\end{minipage} & \begin{minipage}[b]{0.01\columnwidth}\raggedright
country\strut
\end{minipage} & \begin{minipage}[b]{0.56\columnwidth}\raggedright
description\strut
\end{minipage} & \begin{minipage}[b]{0.03\columnwidth}\raggedright
designation\strut
\end{minipage} & \begin{minipage}[b]{0.01\columnwidth}\raggedleft
points\strut
\end{minipage} & \begin{minipage}[b]{0.01\columnwidth}\raggedleft
price\strut
\end{minipage} & \begin{minipage}[b]{0.02\columnwidth}\raggedright
province\strut
\end{minipage} & \begin{minipage}[b]{0.02\columnwidth}\raggedright
region\_1\strut
\end{minipage} & \begin{minipage}[b]{0.01\columnwidth}\raggedright
region\_2\strut
\end{minipage} & \begin{minipage}[b]{0.03\columnwidth}\raggedright
variety\strut
\end{minipage} & \begin{minipage}[b]{0.01\columnwidth}\raggedright
winery\strut
\end{minipage}\tabularnewline
\midrule
\endhead
\begin{minipage}[t]{0.00\columnwidth}\raggedleft
0\strut
\end{minipage} & \begin{minipage}[t]{0.01\columnwidth}\raggedright
US\strut
\end{minipage} & \begin{minipage}[t]{0.56\columnwidth}\raggedright
This tremendous 100\% varietal wine hails from Oakville and was aged
over three years in oak. Juicy red-cherry fruit and a compelling hint of
caramel greet the palate, framed by elegant, fine tannins and a subtle
minty tone in the background. Balanced and rewarding from start to
finish, it has years ahead of it to develop further nuance. Enjoy
2022--2030.\strut
\end{minipage} & \begin{minipage}[t]{0.03\columnwidth}\raggedright
Martha's Vineyard\strut
\end{minipage} & \begin{minipage}[t]{0.01\columnwidth}\raggedleft
96\strut
\end{minipage} & \begin{minipage}[t]{0.01\columnwidth}\raggedleft
235\strut
\end{minipage} & \begin{minipage}[t]{0.02\columnwidth}\raggedright
California\strut
\end{minipage} & \begin{minipage}[t]{0.02\columnwidth}\raggedright
Napa Valley\strut
\end{minipage} & \begin{minipage}[t]{0.01\columnwidth}\raggedright
Napa\strut
\end{minipage} & \begin{minipage}[t]{0.03\columnwidth}\raggedright
Cabernet Sauvignon\strut
\end{minipage} & \begin{minipage}[t]{0.01\columnwidth}\raggedright
Heitz\strut
\end{minipage}\tabularnewline
\bottomrule
\end{longtable}

As you can see, the data contains information on the \texttt{country}
the wine is from, as well as the \texttt{province}, the macro-level
\texttt{region} (which is below the \texttt{province}) and the
micro-level \texttt{region\_2} (which is below the \texttt{region\_1}).
The \texttt{designation} represents the part of the vinyard where the
vintner claims the wine comes from. The \texttt{price} is the price (in
dollars) of the wine. The \texttt{points} are the rating of the wine,
given by \emph{WineMag}'s judges. Finally, the \texttt{variety} reflects
the grape (or grapes) used to make the wine, and the \texttt{winery}
reflects the name of the winery that grew the wine.

\hypertarget{exploration-visualization}{%
\section{1 Exploration \&
Visualization}\label{exploration-visualization}}

This section, please use tidy transformation, like \texttt{group\_by}
and \texttt{summarize}, to obtain the answers to these questions. \#\#
1.1

What is your best guess about the price and rating of wines from each
country?

\hypertarget{section}{%
\subsection{1.2}\label{section}}

What is the highest rated wine in each country?

\hypertarget{section-1}{%
\subsection{1.3}\label{section-1}}

In terms of the points-per-dollar, which country has the best
value-for-money in their wines?

\hypertarget{section-2}{%
\subsection{1.4}\label{section-2}}

\href{https://www.youtube.com/watch?v=Nvxwf1jxdaM}{Orson Welles
disagrees with your friend, and thinks that the US is almost as good as
France at making champagne\ldots but he was paid a lot in free US wine
to say so!}

Does your dataset agree? Are American champagnes rated as highly as
French champagnes?\footnote{Remember: the \texttt{variety} column
  contains the kind of grape that made the wine. And, most Champagne
  wines are actually \texttt{Champagne\ Blend}, when you're as pedantic
  as the typical \emph{WineMag} reader 😁}

\hypertarget{challenge}{%
\subsection{\texorpdfstring{1.5
\textbf{challenge}}{1.5 challenge}}\label{challenge}}

What is the most common wine in each country?

\hypertarget{section-3}{%
\section{2}\label{section-3}}

Fit a statistical model predicting \texttt{points} using
\texttt{country}.

\hypertarget{section-4}{%
\subsection{2.1}\label{section-4}}

Which country has the best wines?

\hypertarget{section-5}{%
\subsection{2.2}\label{section-5}}

Is this country statistically significantly different from the other
countries?

\hypertarget{section-6}{%
\section{3}\label{section-6}}

Look at the plot showing the relationship between price and quality for
each country in the data.

\begin{Shaded}
\begin{Highlighting}[]
\KeywordTok{ggplot}\NormalTok{(wines, }\KeywordTok{aes}\NormalTok{(}\DataTypeTok{x=}\NormalTok{points, }\DataTypeTok{y=}\NormalTok{price)) }\OperatorTok{+}\StringTok{ }
\StringTok{  }\KeywordTok{geom_point}\NormalTok{(}\DataTypeTok{alpha=}\NormalTok{.}\DecValTok{5}\NormalTok{, }\DataTypeTok{lwd=}\DecValTok{0}\NormalTok{) }\OperatorTok{+}\StringTok{ }
\StringTok{  }\KeywordTok{geom_smooth}\NormalTok{(}\DataTypeTok{method=}\NormalTok{lm) }\OperatorTok{+}\StringTok{ }
\StringTok{  }\KeywordTok{facet_grid}\NormalTok{(}\OperatorTok{~}\NormalTok{country) }
\end{Highlighting}
\end{Shaded}

\begin{verbatim}
## `geom_smooth()` using formula 'y ~ x'
\end{verbatim}

\begin{figure*}
\includegraphics{04-final_files/figure-latex/unnamed-chunk-2-1} \end{figure*}

\hypertarget{section-7}{%
\section{3.1}\label{section-7}}

From the scatterplot above, does the relationship between price and
quality appear to be constant over countries? Does the relationship
appear to be linear?

\hypertarget{section-8}{%
\section{3.2}\label{section-8}}

Fit a multilevel model that allows the slope and intercept to
vary.\footnote{Remember to center your X variables when running
  multilevel models! This improves the numerical stability of the
  estimating algorithms, and makes it easier to interpret \(\alpha\) as
  ``the mean of \(y\) when \(x\) is at its mean.''}

\hypertarget{section-9}{%
\subsection{3.3}\label{section-9}}

In the model from 3.2, which country has the lowest baseline price for
an average wine? Is it statistically significantly different from the
other countries?

\hypertarget{section-10}{%
\subsection{3.4}\label{section-10}}

In the model from 3.2, which country pays the highest ``premium'' for
higher-rated wine?\footnote{Another way of phrasing this is: which
  country's wines gain the highest price each extra point of rating?} Is
this statistically significantly different from the other countries?

\hypertarget{challenge-1}{%
\subsection{\texorpdfstring{3.5
\textbf{Challenge}}{3.5 Challenge}}\label{challenge-1}}

Your friend says they bought a 100-point wine for about \$70, but are
not sure where it is from.

You use your model model to predict the price of a 100-point wine in
every country\footnote{Remember that you may have centered your
  variables! 100 points is \emph{not} the same as 100 points above the
  mean rating!}, and find that it might have come from one of two
countries. Given what you have done in Section 1 and here, which country
do you think is more likely and why?

\hypertarget{section-11}{%
\section{4}\label{section-11}}

Using logistic regression\ldots{}

\hypertarget{section-12}{%
\subsection{4.1}\label{section-12}}

Fit a model predicting whether a wine is from France using the wine's
price and rating. In this model,

\hypertarget{section-13}{%
\subsubsection{4.1.1}\label{section-13}}

Do more expensive wines tend to be from France?

\hypertarget{section-14}{%
\subsubsection{4.1.2}\label{section-14}}

Do more highly-rated wines tend to be from France?

\hypertarget{section-15}{%
\subsubsection{4.1.3}\label{section-15}}

Using a cutoff of .2, show me the confusion matrix. How many French
wines does our model classify as ``not French''?

\hypertarget{section-16}{%
\subsubsection{4.2.1}\label{section-16}}

Do more expensive wines tend to be from the US?

\hypertarget{section-17}{%
\subsubsection{4.2.2}\label{section-17}}

Do more highly-rated wines tend to be from the US?

\hypertarget{challenge-2}{%
\subsection{\texorpdfstring{4.2
\textbf{Challenge}}{4.2 Challenge}}\label{challenge-2}}

Which model is more accurate, the model predicting French wine, or the
model predicting American wine?



\end{document}
